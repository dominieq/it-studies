\documentclass[11pt]{article}

\usepackage[T1]{fontenc}
\usepackage[english]{babel}
\usepackage[utf8]{inputenc}
\usepackage{lmodern}
\selectlanguage{english}

\title{Elements of Computational Intelligence \\ Data Envelopment Analysis \\ Case study report}
\author{Dominik Szmyt 132326, Angelika Szyszka 132328 \\ Monday, 11:45}
\date{}

\usepackage[margin=1.25in]{geometry}
\usepackage{calc}
\usepackage{graphicx}
\usepackage[dvipsnames]{xcolor}
\usepackage{subcaption}
\usepackage{float}
\usepackage{enumitem}
\usepackage{ulem}
\usepackage{contour}
\usepackage{tabu}
\usepackage{listings}

\setlist[description]{labelwidth=18.61502pt, labelsep=0pt, itemindent=0pt}
\newcommand{\myDot}{\color{blue}{>>}}

\renewcommand{\ULdepth}{1.8pt}
\contourlength{0.8pt}

\newcommand{\myuline}[1]{%
  \uline{\phantom{#1}}%
  \llap{\contour{white}{#1}}%
}

\tabulinesep = 3mm
%
%
\begin{document}
\maketitle
\section{Analyzed problem}
Given is a set of 18 countries for which we need to evaluate their scientific wealth. They are evaluated using 3 inputs and 2 outputs descibled below:
\newline \newline
\textbf{Inputs:}
\begin{description}
	\item[i1] - GDP (in 109 dollars)
	\item[i2] - active population (in millions)
	\item[i3] - R\&D expenditure (in 106 dollars) 
\end{description}
\textbf{Outputs:}
\begin{description}
	\item[o1] - number of publications in the SCI (1993)
	\item[o2] - number of patents granted by the European Patent OfficeS(1993)
\end{description}
%
%
\section{Chosen model}
The problem stated above has been solved using Charnes, Cooper and Rhones (CCR) Data Envelopment Analysis model.
%
\pagebreak
%
\section{Efficiency scores}
Efficiency scores obtained by model for all countries are presented in the table below.
\begin{center}
	\begin{tabular}{| m{6cm} | m{6cm} |}
	\hline
	\textbf{Country} & \textbf{Efficiency} \\
	\hline \hline
	Austria & 0.9801283\\
	\hline
	Belgium & 0.3644829\\
	\hline
	Denmark & 0.4627883\\
	\hline
	Finland & 0.3308322\\
	\hline
	France & 0.5301964\\
	\hline
	Germany & 1\\
	\hline
	Ireland & 0.458212\\
	\hline
	Italy & 0.3952348\\
	\hline
	Netherlands & 0.9138391\\
	\hline
	Norway & 0.2287916\\
	\hline
	Spain & 0.08821843\\
	\hline
	Sweden & 0.4305553\\
	\hline
	Switzerland & 1\\
	\hline
	U. Kingdom & 0.3912678\\
	\hline
	Australia & 0.1388172\\
	\hline
	Canada & 0.1381905\\
	\hline
	Japan & 0.3504877\\
	\hline
	USA & 0.2752869\\
	\hline
	\end{tabular}
\end{center}
%
%
\section{Efficient units}
There are 2 countries that perform efficiently:
\begin{itemize}
	\item Germany
	\item Switzerland
\end{itemize}
%
\pagebreak
%
\section{Inefficient units}
There are 16 inefficient countries. To make them efficient they need to reduce their inputs by the following values:
\begin{center}
	\begin{tabular}{| m{3cm} | m{3cm} | m{3cm} | m{3cm} |}
	\hline
	\textbf{Country} & \textbf{GDP} & \textbf{Population} & \textbf{R\&D} \\
	\hline \hline
	Austria & 3.656389 & 77.10211 & 39.24657 \\
	\hline
	Belgium & 139.1782 & 2692.686 & 2026.664 \\
	\hline
	Denmark & 76.28406 & 1559.525 & 1185.089 \\
	\hline
	Finland & 72.9393 & 1674.258 & 1519.68 \\
	\hline
	France & 622.02 & 11925.97 & 13580.61 \\
	\hline
	Ireland & 27.0894 & 744.9585 & 161.9946 \\
	\hline
	Italy & 740.2326 & 14036.6 & 8581.618 \\
	\hline
	Netherlands & 27.5715 & 604.0744 & 480.8642 \\
	\hline
	Norway & 87.14655 & 1658.869 & 1504.628 \\
	\hline
	Spain & 433.0962 & 14315.88 & 3745.599 \\
	\hline
	Sweden & 139.514 & 2429.251 & 3944.543 \\
	\hline
	U. Kingdom & 638.5601 & 17362.87 & 12871.64 \\
	\hline
	Australia & 253.1877 & 7578.409 & 3701.364 \\
	\hline
	Canada & 486.0605 & 12845.27 & 7414.147 \\
	\hline
	Japan & 2375.916 & 43160.09 & 67253.75 \\
	\hline
	USA & 4314.217 & 96005.65 & 100919.9 \\
	\hline
	\end{tabular}
\end{center}
\end{document}