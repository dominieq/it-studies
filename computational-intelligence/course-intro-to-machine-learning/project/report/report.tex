\documentclass[11pt]{article}

\usepackage[T1]{fontenc}
\usepackage[polish]{babel}
\usepackage[utf8]{inputenc}
\usepackage{lmodern}
\selectlanguage{polish}

\title{Autoenkoder do kolorowania obrazów czarno-białych}
\author{Dominik Szmyt 132326}
\date{2020-02-08}

\usepackage{graphicx}
\usepackage{float}
\usepackage{indentfirst}
\usepackage[margin=1.25in]{geometry}
\usepackage[dvipsnames]{xcolor}
\usepackage{enumitem}
\usepackage{tabu}

\tabulinesep = 3mm

\begin{document}
\maketitle
\section{Opis problemu}
Głównym problemem jest stworzenie autoenkodera, który po nauczeniu, będzie w stanie pokolorować czarno-biały obraz.

\section{Sposób rozwiązania problemu}
Korzystam z danych ze zbiory cifar101. Zbiór treningowy i testowy przekształcam do odcieni szarości. Następnie tworzę autoenkoder.

Enkoder składa się z trzech warstw konwolucyjnych o rozmiarach filtrów 64, 128 i 256. Dekoder składa się z czterech transponowanych warstw konwolucyjnych o rozmiarach filtrów odpowiednio 256, 128 i 64. Ostatni rozmiar filtra jest równy ilości kanałów w zbiorze treningowym

Użyta przeze mnie funkcja straty to mean squared error.
\section{Analiza osiągniętych wyników}
Sieć po trzech epokach nauczyła się kolorować niektóre elementy na brązowo. Jest to kolor najbliższy wielu innym barwom stąd właśnie pojawił się on tak szybko.

\end{document}